\begin{frame}{Modelling}{Introduction}
	First principles model derived, linearised and further developed for observer based control.
	\begin{itemize}
		\item \textbf{Main goal:} Capture dynamics important for the box air temperature (main control objective)
%		\begin{itemize}
%			\item Thermal masses: Trailer box, cargo, evaporator, condenser.
%		\end{itemize}
		\item \textbf{Modular approach:} Inputs, outputs, internal equations/states
		\item \textbf{Slow vs. fast dynamics:} States or algebraic equations	
		\item Enthalpies are defined in steady-state (SS)
	\end{itemize}\bigskip
Only the most relevant components and equations are included in the following.
\end{frame}



%%%%%%%%%%%%%%%%%

%\begin{frame}{Modelling}{Component Models}
%	Each component is modelled separately. The components are:
%	\begin{itemize}
%		\item Expansion Valve
%		\item Pipe joining junction
%		\item Compressor
%		\item Condenser
%		\item Flash tank
%		\item Evaporator
%		\item Box
%	\end{itemize}
%	
%	Only the most relevant components and equations are included in the following.
%\end{frame}



%%%%%%%%%%%%%%%%%

\begin{frame}{Modelling}{Expansion Valve}
	\textbf{Purpose:} Lower pressure of liquid refrigerant from flash tank to evaporator.
	\begin{itemize}
		\item Adiabatic process: Modelled algebraicly
	\end{itemize}
	Flow through valve:
	\begin{equation} \label{eq:ExpansionValve_Alt}
		\begin{split}
			\dot{m} & = f_p(\Theta) K  \sqrt{\frac{1}{v_{in}} (p_{in} - p_{out})} \\
		\end{split}
	\end{equation}

\end{frame}



%%%%%%%%%%%%%%%%%

\begin{frame}{Modelling}{Compressor}
	\textbf{Purpose:} Generate pressure sufficient for refrigerant flow and condenser heat flow to ambient air.
	\begin{itemize}
		\item Two compressor stages
		\item Modelled as reciprocating (piston) compressor
		\item Modelled algebraicly - Fast dynamics do not affect slow evaporator dynamics
	\end{itemize}
	Flow through compressor:
	\begin{equation}
		\dot{m} = \left(\frac{V_1}{v_1} - \frac{V_C}{v_2}\right) \frac{\omega}{2} \label{eq:comp_mass_flow}
	\end{equation}
\end{frame}

%%%%%%%%%%%%%%%%%

\begin{frame}{Modelling}{Condenser}
	\textbf{Purpose:} Exchange heat from hot vapour refrigerant to outside ambient air.
	\begin{itemize}
		\item Modelled as one control volume (CV)
		\item States: $M_r$ and $T_m$
	\end{itemize}
	\begin{figure}[h!]
		\centering
		\includegraphics[width=1\textwidth]{../Graphics/Condenser.pdf}
		\caption{Diagram of condenser control volume}
		\label{fig:condenser_CV}
	\end{figure}
\end{frame}
\begin{frame}{Condenser}{Equations}
	CV energy balance:
	\begin{equation}
		h_{out} = h_{in} - \frac{Q_{rm}}{\dot{m}_{in}} \label{eq:Condenser_Enthalpy}
	\end{equation}
	CV mass balance:
	\begin{equation}
		\frac{dM_r}{dt} 	 = \dot{m}_{in}(t) - \dot{m}_{out}(t) \label{eq:Condenser_ChangeOfMass}
	\end{equation}
	Metal energy balance:
	\begin{equation}
		\frac{dT_m}{dt} 	 = \frac{Q_{rm} - Q_{ma}}{M_m \cdot Cp_m} \label{eq:Condenser_ChangeOfTemperature}
	\end{equation}
\end{frame}




%%%%%%%%%%%%%%%%%

\begin{frame}{Modelling}{Evaporator}
	\textbf{Purpose:} Exchange heat from box air to cold liquid refrigerant.
	\begin{itemize}
		\item Superheat is important control objective
		\begin{itemize}
			\item Liquid slugging
			\item Efficiency
		\end{itemize}
		\item Modelled with two CVs: liquid-vapour and vapour separated by $\sigma$
		\item States: $T_{mlv}$, $T_{mv}$, $M_{lv}$, $M_{v}$ and $T_{v}$
	\end{itemize}
	
	\vspace{0.4cm}
	\begin{figure}[h!]
		\centering
		\includegraphics[width=1\textwidth]{../Graphics/Evaporator_CV_diagram.pdf}
		\label{fig:evap_CV}
	\end{figure}

\end{frame}

\begin{frame}{Evaporator}{Equations}
	Moving boundary:
	\begin{equation}
		\sigma = \frac{M_{lv} \cdot v_{lv}}{V_i} \label{eq:Evaporator_boundary}
	\end{equation}
	Liquid-vapour CV metal energy balance:
	\begin{equation}
		\frac{dT_{mlv}}{dt}  = \frac{Q_{amlv}-Q_{mlv} + Q_{mvmlv}}{M_m \cdot \sigma \cdot Cp_m} \label{eq:evap_dT_ml}
	\end{equation}
	Vapour CV metal energy balance:
	\begin{equation}
		\frac{dT_{mv}}{dt} = \frac{Q_{amv} - Q_{mv} - Q_{mvmlv}}{M_m \cdot (1 - \sigma) \cdot Cp_m} \label{eq:evap_dT_mv}
	\end{equation}
	Refrigerant mass balances:
	\begin{equation} \label{eq:evap_dMlv}
		\frac{dM_{lv}}{dt} = \dot{m}_{in} - \dot{m}_{dew} \hspace{0.8cm}  \frac{dM_v}{dt} = \dot{m}_{dew} - \dot{m}_{out}
	\end{equation}
	Vapour refrigerant (output) temperature:
	\begin{equation}\label{eq:tv_initial}
		\frac{dT_{v}}{dt} = \bar{T_v} - T_v
	\end{equation}

\end{frame}

%%%%%%%%%%%%%%%%%

\begin{frame}{Modelling}{Box}
	\textbf{Purpose:} Isolate cargo from ambient air and allow for air circulation.
	\begin{itemize}
		\item Contains greatest thermal masses: Box and Cargo
		\item Cargo coupled to air due to large cargo surface area
		\item States: $T_{air}$, $T_{box}$ and $T_{cargo}$
	\end{itemize}
	\begin{figure}[h!]
		\centering
		\includegraphics[width=0.8\textwidth]{../Graphics/Box.pdf}
		\caption{Simplified diagram of trailer box}
		\label{fig:box_diagram}
	\end{figure}


\end{frame}

\begin{frame}{Box}{Equations}
	Air, box and cargo energy balances:
	\begin{align}
		\frac{dT_{air}}{dt} &= \frac{Q_{ca} + Q_{ba} + Q_{fan} -Q_{cool}}{M_{air} \cdot Cp_{air}} \label{eq:box_dT_air} \\
		\frac{dT_{box}}{dt} &= \frac{Q_{amb} - Q_{ba}}{M_{box} \cdot Cp_{box}} \label{eq:box_dT_box} \\
		\frac{dT_{cargo}}{dt} &= \frac{-Q_{ca}}{M_{cargo} \cdot Cp_{cargo}} \label{eq:box_dT_cargo}
	\end{align}
	Heat flows:
	\begin{align}
		Q_{cool}   & = Cp_{air} \cdot \dot{m}_{air} \cdot (T_{ret} - T_{sup})	\label{eq:box_Qcool} \\
		Q_{amb}    & = (T_{ambi} - T_{box}) \cdot U A_{amb}						\label{eq:box_Qab}   \\
		Q_{ba}     & = (T_{box} - T_{air}) \cdot U A_{ba}						\label{eq:box_Qba}   \\
		Q_{ca}     & = (T_{cargo} - T_{air}) \cdot U A_{ca}                  	\label{eq:box_Qca}
	\end{align}
\end{frame}



%%%%%%%%%%%%%%%%%

\begin{frame}{Modelling}{Component Input/Output relationship}
	\begin{itemize}
		\item \color{red} Red \color{black}: Missing outputs - Steady-state (SS) values from Hi-Fi simulation is used
		\item \color{blue} Blue \color{black}: Unused outputs of components
	\end{itemize}
	\begin{figure}[h!]
		\centering
		\includegraphics[width=1.04\textwidth]{../Graphics/Block_Diagram_inout_flowValveVersion.pdf}
		\label{fig:Block_diagram_inout}
	\end{figure}
	
\end{frame}

%%%%%%%%%%%%%%%%%

\begin{frame}{Modelling}{Component Input/Output relationship}
	\begin{itemize}
		\item All table lookups are SS variables at operating point (OP)
			\begin{itemize}
				\item Reduces complexity and eases linearisation
				\item Downside: Removes dependencies between states and inputs
			\end{itemize}
	\end{itemize}
\end{frame}




%%%%%%%%%%%%%%%%%

\begin{frame}{Modelling}{Formulating non-linear state-space model}
	A non-linear state space model is formulated
	\begin{itemize}
		\item State equations are collected in a $f(x,u,d)$ vector
		\item Algebraic equations are constraints
	\end{itemize}

	The state, input and disturbance vector are:
	\begin{equation}  \label{eq:xu}
		\begin{split}
			x & = [M_{pjj}	\;
				M_{con} \;
				T_m \;
				\dot{m}_{air}\;
				T_{mlv}      \;
				T_{mv}       \;
				M_{lv}       \;
				M_v          \;
				T_{air}      \;
				T_{box}      \;
				T_{cargo}    \;
				T_v]^T \\
			u & = [\omega	\;
				\Theta_1	\;
				\Theta_2     \;
				U_{fan_1}    \;
				U_{fan_2}]^T \\
			d & = [T_{ambi}]
		\end{split}
	\end{equation}
	System contains 12 states - a lot less than previous work.

\end{frame}



%%%%%%%%%%%%%%%%%

\begin{frame}{Modelling}{Non-linear state-space system}
	The resulting non-linear states-space system becomes:
	\begin{figure}[h!]
		\centering
		\includegraphics[width=1.04\textwidth]{Graphics/f_x_u.jpeg}
		\label{fig:f_x_u}
	\end{figure}
\end{frame}





%%%%%%%%%%%%%%%%%

\begin{frame}{Modelling}{Verification}
	\begin{itemize}
		\item Model is verified with simulation
		\item Simulated at OP derived from Hi-Fi simulation
		\item 'Faulty' states: $M_{pjj}$ and $M_{Con}$ with constant derivatives
		\begin{itemize}
			\item Not affecting other states $\rightarrow$ Removed
		\end{itemize}
	\end{itemize}
	\begin{figure}[h]
		\centering
		\includegraphics[width=0.9\textwidth]{../Graphics/nonlin_sim_faulty_Mass.png}
		\label{fig:non_lin_sim_faulty_Mass}
	\end{figure}
\end{frame}




%%%%%%%%%%%%%%%%%

\begin{frame}{Modelling}{Verification: $M_{lv}$, $M_{v}$ and $\dot{m}_{air}$}
	\begin{itemize}
		\item States plotted: $M_{lv}$, $M_v$ and $\dot{m}_{air}$
		\item Mass of vapour inside evaporator $M_v$ has constant non-zero derivative at SS
		\item Considered numeric error and mitigated by subtracting function of compressor speed
	\end{itemize}
	\begin{figure}[h]
		\centering
		\includegraphics[width=1\textwidth]{../Graphics/nonlin_sim_Mass_Mvfucked.png}
		\caption{Notice two y-axis}
		\label{fig:non_lin_sim_Mass_Mvfucked}
		
	\end{figure}
	
\end{frame}



%%%%%%%%%%%%%%%%%

\begin{frame}{Modelling}{Verification: Discussion}
	\begin{itemize}
		\item Likely cause of errors: Simplifications in modelling section
		\begin{itemize}
			\item Pressure inside components not defined as states
			\item Results in mass inside components not affecting pressure and thereby flow through component
		\end{itemize}
	\end{itemize}

\end{frame}



%%%%%%%%%%%%%%%%%

\begin{frame}{Modelling}{Linearisation}
	\begin{itemize}
		\item Motivation: Non-linear model is not directly applicable for control strategy 
		\item Linearisation is thus performed at relevant OP
	\end{itemize}
	Non-linear model is transformed to the following form:
	\begin{equation} \label{eq:state_space}
		\begin{split}
			\dot{x} & = Ax + Bu + B_dd \\
			y 		& = Cx
		\end{split}
	\end{equation}
	by using the first order taylor expansion:
	\begin{equation}
		f(x) \approx f(x_o) + \left. \dfrac{\partial f}{\partial x} \right |_{x_o} \cdot (x-x_0)		
	\end{equation}
	where $f(x)$ = $\dot{x}$ and $x = (x, u, d)$.


\end{frame}



%%%%%%%%%%%%%%%%%

%\begin{frame}{Modelling}{Linearisation}
%	Resulting full taylor expansion expression:
%	\begin{equation} \label{eq:taylor}
%		\begin{split}
%			\dot{x}   \approx   f(x_o, u_o, d_o)   & +
%			\left. \dfrac{\partial f}{\partial x} \right |_{x_o, u_o, d_o} \cdot (x-x_0)  + 
%			\left. \dfrac{\partial f}{\partial u} \right |_{x_o, u_o, d_o} \cdot (u-u_0) \\ & +
%			\left. \dfrac{\partial f}{\partial d} \right |_{x_o, u_o, d_o} \cdot (d-d_0)
%		\end{split}
%	\end{equation}
%	With each partial derivative yielding a Jacobian on the form:
%	\begin{equation} \label{eq:jacobians}
%		\dfrac{\partial f}{\partial x} =
%		\begin{bmatrix}
%			\dfrac{\partial f_1}{\partial x_1} & \cdots & \dfrac{\partial f_1}{\partial x_{nx}} & \\
%			\vdots & \ddots & \vdots & \\
%			\dfrac{\partial f_e}{\partial x_1} & \cdots & \dfrac{\partial f_e}{\partial x_{nx}} &
%		\end{bmatrix}
%	\end{equation}
%	Where derivatives are made with respect to both states, inputs and disturbance $(x,u,d)$
%	
%
%\end{frame}
%
%
%
%%%%%%%%%%%%%%%%%%
%
%\begin{frame}{Modelling}{}
%	Resultingly the system matrix, input and disturbance input matrices are:
%	\begin{equation}
%		\left. \dfrac{\partial f}{\partial x} \right |_{x_o, u_o, d_o} = A, \;\;
%		\left. \dfrac{\partial f}{\partial u} \right |_{x_o, u_o, d_o} = B, \;\;
%		\left. \dfrac{\partial f}{\partial d} \right |_{x_o, u_o, d_o} = B_d
%	\end{equation}
%	Notes regarding operating point:
%	\begin{itemize}
%		\item Should be at equilibrium
%		\item If not then some eigenvalues of $A$ will have zero real parts. 
%		\item In this case not all dynamics of the system are captured at that operating point (Hartman-Grobman theorem)
%	\end{itemize}
%\end{frame}



%%%%%%%%%%%%%%%%%

\begin{frame}{Modelling}{Linearisation: B matrix}
	\begin{itemize}
		\item $A$ matrix is stable
		\item One pole is -1.3e-9 indicating a zero pole
		\begin{itemize}
			\item Inferred: Not all dynamics are captured by the chosen operating point
		\end{itemize}

		\item Resulting B matrix shows two inputs not mapping to states. These are simply removed.
	\end{itemize}
	\begin{align}
		B = &\kbordermatrix{
			& \omega & \Theta_{eva} & \Theta_{con} & U_{fan\_con} & U_{fan\_eva} \\
			T_m 			& 0 & 0 & 0 & -0.0093 & 0\\
			\dot{m}_{air}	& 0 & 0 & 0 & 0 & \text{7.9630e-04}\\
			T_{mlv}			& 0 & 0 & 0 & 0 & 0\\
			T_{mv}			& 0 & 0 & 0 & 0 & 0.0011\\
			M_{lv}			& 0 & \text{5.0703e-04} & 0 & 0 & 0\\
			M_v 			& 0 & 0 & 0 & 0 & 0\\
			T_{air}  		& 0 & 0 & 0 & 0 & \text{1.0069e-04}\\
			T_{box}	 		& 0 & 0 & 0 & 0 & 0\\
			T_{cargo} 		& 0 & 0 & 0 & 0 & 0\\
			T_v 			& 0 & 0 & 0 & 0 & 0
		} \label{eq:B_full}
	\end{align}

\end{frame}



%%%%%%%%%%%%%%%%%

%\begin{frame}{Modelling}{Linearisation: Discussion}
%	\begin{itemize}
%		\item $\omega$ and $\Theta_2$ do not map to any states
%		\begin{itemize}
%			\item Explanation: Too simple model - several components do not connect
%			\item Example: No flow coupling between $\dot{m}_{in}$ and $\dot{m}_{out}$ of condenser.
%			\begin{itemize}
%				\item Result: No control of flow into evaporator
%			\end{itemize}
%		\end{itemize}
%		\item $A$ matrix is stable
%		\item One pole is -1.3e-9 indicating a zero pole
%		\begin{itemize}
%			\item Inferred: Not all dynamics are captured by the chosen operating point
%		\end{itemize}
%	\end{itemize}
%\end{frame}



%%%%%%%%%%%%%%%%%

\begin{frame}{Modelling}{Controllability and Observability}
	\textbf{Controllability}
	\begin{itemize}
		\item Checked to ensure that all inputs map to states
		\item Controllability matrix should be full rank:
	\end{itemize}
	\begin{equation} \label{eq:ctrb}
		Q_c = [A|B] = \begin{bmatrix}  B & AB & \cdots & A^{n-1}B  \end{bmatrix}
	\end{equation}

	\textbf{Observability}
	\begin{itemize}
		\item Checked to ensure that all states map to outputs
		\item Observability matrix should be full rank:
	\end{itemize}
	\begin{equation}
		Q_o = [A|C] = \begin{bmatrix}
			C \\ CA \\ \vdots \\ CA^{n-1}
		\end{bmatrix}
	\end{equation}
\end{frame}



%%%%%%%%%%%%%%%%%

\begin{frame}{Modelling}{Controllability and Observability: Results}
	\begin{itemize}
		\item Controllability matrix has full rank: Rank of [A|B] = 10
		\item Observability matrix has \underline{not} full rank: Rank of [A|C] = 8
			\begin{itemize}
				\item Actions are taken to mitigate: Kalman decomposition for unobservable system
				\item Two states are not observable: 
				\begin{itemize}
					\item Condenser metal temperature: $T_m$
					\item Evaporator vapour CV mass: $M_v$
				\end{itemize} 
			\end{itemize}
		
	\end{itemize}
\end{frame}



%%%%%%%%%%%%%%%%%

%\begin{frame}{Modelling}{Kalman decomposition for unobservable system}
%	\begin{itemize}
%		\item Used to obtain an observable subsystem
%	\end{itemize}
%	For the Kalman decomposition, there exists a nonsingular P  $\in \mathbb{R} ^{n x n}$ such that
%	\begin{equation}
%		PAP^{-1} = \begin{bmatrix}
%			A_{11}       & 0 \\
%			A_{21}       & A_{22} \\
%		\end{bmatrix}
%	\end{equation}
%	
%	and
%	
%	\begin{equation}
%		CP^{-1} = \begin{bmatrix}
%			C_{1}       & 0 \\
%		\end{bmatrix}
%	\end{equation}
%	
%	where $A_{11} \in \mathbb{R} ^{l x l}$ and $C_{1} \in \mathbb{R} ^{p x l}$, where p is the number of outputs and l is the number of observable states.
%	
%	Input and disturbance matrices are transformed as such:
%	\begin{equation}
%		PB = \begin{bmatrix}
%			B_1 \\
%			B_2
%		\end{bmatrix}, \
%		PB_d = \begin{bmatrix}
%			{B_d}_1 \\
%			{B_d}_2
%		\end{bmatrix}
%	\end{equation}
%\end{frame}



%%%%%%%%%%%%%%%%%

\begin{frame}{Modelling}{Kalman Decomposition reduced system}
	Resulting B matrix:
	\begin{align}
		B_1 & = \kbordermatrix{
			& \omega & \Theta_{eva} & \Theta_{con} & U_{fan\_con} & U_{fan\_eva} \\
			\dot{m}_{air}	& 0 & 0 & 0 & 0 & 0.00083\\
			T_{mlv}			& 0 & 0 & 0 & 0 & 0\\
			T_{mv}			& 0 & 0 & 0 & 0 & 0.0011\\
			M_{lv}			& 0 & 0.0005 & 0 & 0 & 0\\
			T_{air}  		& 0 & 0 & 0 & 0 & 0.0001\\
			T_{box}	 		& 0 & 0 & 0 & 0 & 0\\
			T_{cargo} 		& 0 & 0 & 0 & 0 & 0\\
			T_v 			& 0 & 0 & 0 & 0 & 0
		} \label{eq:B1}
	\end{align}
	\begin{itemize}
		\item Another input $U_{fan\_con}$ does not map to states
		\item Is also removed
	\end{itemize}
\end{frame}



%%%%%%%%%%%%%%%%%

%\begin{frame}{Modelling}{Condition number}
%	\begin{itemize}
%		\item Brings nuance to degree of controllability and observability
%		\item "Large" condition numbers are generally undesirable
%	\end{itemize}
%	Definition:
%	\begin{equation}
%		\kappa (G(j\omega)) = \frac{\bar{\sigma}(G(j\omega))}{\underline{\sigma}(G(j\omega))}
%	\end{equation}
%	where $G(j\omega)$ is the transfer matrix of the system and
%	\begin{itemize}
%		\item $\bar{\sigma}$ is the largest singular value
%		\item $\underline{\sigma}$ is the smallest singular value
%	\end{itemize}
%	Found from Singular Value Decomposition (SVD).
%	
%	\begin{itemize}
%		\item Condition numbers for $Q_c$ are:
%		\begin{itemize}
%			\item 1.195e+9 for 'two state reduced' system 
%			\item 2.934e+8 for Kalman decomposition reduced system
%		\end{itemize}
%	\end{itemize}
%\end{frame}


