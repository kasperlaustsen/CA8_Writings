While the work in this project has laid a solid foundation in the development of the MIMO control system, there are still aspects that can be investigated, improved and implemented. This section will discuss some of the areas that future work could involve.\\

Firstly the control model needs improvements in several fields. During the later stages of the project a highly analytical investigation of component models was initialized. Some model errors were discovered during this work, and some of those were investigated in \cref{app:tj_2}. There were unfortunately clear mismatches in the control model, for which the solution was considered too time consuming for the scope of the project. Continuing the investigations of the component models, and resolving issues would be of high priority in future work. \\

In a more matured version of the control model, more of the critical variables would be implemented as states. This would reduce the amount of steady state values hard coded into the model, likely increasing the accuracy of the model by a large margin. The group tried to develop a model with as few states as possible. This was done to limit the complexity of  modeling that can be the result thermodynamic modeling. However, it is apparent that too much dynamic behavior and interconnections was left out of the current model.\\

The main part of the project was spent developing the model, which left little time for the actual controller design. As a result, a LQR controller was designed. While the control strategy itself is promising, it has clear limitations. Firstly, the standard LQR problem is not concerned with disturbances. This proved an issue, as the controller had poor disturbance rejection. Reference tracking and integral action would have been the first control actions taken, had there been more time. That is thus a good starting place for future work on the controller. Secondly, the LQR problem is not well suited for systems with constraints on the input and output signals. In this project there are constraints on the input signals, as the actuators have practical limitations. At the same time, the output signals are bound by the underlying physics, which limit their range and maximum slew rate. As such a more relevant control structure may have been the MPC controller, which is essentially a optimal control problem under constraints.\\

Linearising the model is essential in deriving a state space model, for which to implement linear control strategies on. It does however pose an issue, that the model becomes increasingly more inaccurate the farther from the linearisation point. It would have been interesting to thoroughly investigate the range in which the model is sufficiently accurate. Knowing that, a gain scheduling scheme could be developed, where a different set of feedback gains would be used for different operating regions. 