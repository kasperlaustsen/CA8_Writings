The reefer trailer is an insulated trailer designed and built by Schmidt Cargobull with a cooling system developed by Bitzer. It is attached to a truck and has a length of 13.4 meters. The trailer features room for 33 Euro-pallets of cargo and is used to transport various goods that require specific  conditions during transportation, e.g. temperature, humidity and CO2 level. The cooling system used to control the internal temperature in the trailer is powered by an internal battery. This allows the reefer trailer cooling to be powered independently of the truck.

While a reefer trailer usually has its own cooling system installed, the reefer trailer in this project utilizes a custom HVAC system designed by BITZER. This system is built to facilitate testing of advanced control strategies.

\subsection{Refrigeration Systems}
The role of a refrigeration system resides in the second law of thermodynamics. It can be formulated in many ways, one of which is that heat transfer always happens from a warmer to a colder body but never in the reverse direction. This essentially motivates a refrigeration system as the objective is extract heat from a cold to a warm body, e.g. from cargo in a reefer trailer to the surrounding ambient air.

\subsubsection{Principle of refrigeration}
A textbook example of a refrigeration system is a single stage refrigeration system as depicted in \cref{fig:HVAC_Diagram_std}. It is comprised of 4 components, a compressor, an expansion valve, and two heat exchangers, the evaporator and the condenser. Assuming that the pressure loss in the evaporator and condenser is negligible, there will be a high pressure side and a low pressure side of the refrigeration circuit. The high pressure side accommodates that the refrigerant temperature is increased above the hot body temperature, thus allowing for heat transfer from refrigerant to hot body. On the low pressure side, the refrigerant temperature is less than the cold body, which enables heat transfer from cold body to the refrigerant. In order to further describe the refrigeration process, a pressure-enthalpy diagram (p-h diagram) is consulted.
\begin{figure}[h]
	\centering
	\begin{minipage}{0.5\textwidth}
		\centering
		\includegraphics[width=1\textwidth]{Graphics/HVAC_Diagram_std.pdf} % first figure itself
		\caption{Illustration of single stage refrigeration cycle}
		\label{fig:HVAC_Diagram_std}
	\end{minipage}\hfill
	\begin{minipage}{0.5\textwidth}
		\centering
		\includegraphics[width=1\textwidth]{Graphics/p-h_diagram_std} % second figure itself
		\caption{Illustration of p-h diagram of single stage refrigeration cycle}
		\label{fig:p-h_diagram_std}
	\end{minipage}
\end{figure}
The pressure and enthalpy of the refridgerant as it moves around in the refrigeration cycle are illustrated in \cref{fig:p-h_diagram_std}. The four connection points between the components from \cref{fig:HVAC_Diagram_std} are illustrated in \cref{fig:p-h_diagram_std}. The transitions from each of the points will be described below:

\begin{itemize}
	\item From point 1 to 2 the compressor increases the pressure (and enthalpy) of the vapor refrigerant. The increase in enthalpy is not desired, but cannot be omitted as it is not an isenthalpic process. The compressor also drives the flow of refrigerant in the circuit.
	\item From point 2 to 3 the high pressure and high temperature vapor refrigerant runs through the condenser and heat is transferred to the ambient air (hot body). This is an isobaric exothermic process. The refrigerant is intended to change phase from vapor to liquid, hence the name condenser.
	\item From point 3 to 4, the high pressure but low enthalpy liquid refrigerant passes through an isenthalpic expansion, which lowers the pressure and thereby also the temperature. Some of the liquid may change phase to gas (called flash gas), s.t. the refrigerant is now a mix of vapor and liquid.
	\item From point 1 to 4, the low pressure and low temperature liquid-vapor refrigerant runs through the evaporator and heat is transferred from the load to the refrigerant. The refrigerant completely changes phase to vapor. This is an isobaric endothermic process. 
\end{itemize}
 Now that the principle of refrigeration is introduced, the investigation of the refrigeration system of this project can commence.
 
\subsection{The BITZER Refrigeration System}
The refrigeration system used in this project contains more components than the textbook example. As seen in \cref{fig:HVAC_Diagram}, the compressor is comprised of two stages, rather than one. Similarly two valves are being used. The valves are separated by a flash tank, that enables transfer of flash gas from the condenser throttle valve to the second stage of the compressor. The removed flash gas ensures that the liquid refrigerant that enters the evaporator has lower enthalpy resulting in a higher cooling capacity. This enables higher efficiency of the system than a single stage system.\\
The subcooling throttle valve is only used for debugging purposes, and is fully open at all times in this project. The reefer system has two fans that enables variable amounts of forced convection instead of natural convection. The heat transfer coefficients from the heat exchangers (evaporator and condenser) to the cold and hot body can such be varied with the fan speed. Additionally the airflow circulation flow inside the trailer is controlled by the evaporator fan.

\begin{figure}[h]
	\centering
	\begin{minipage}{0.6\textwidth}
		\centering
		\includegraphics[width=1\textwidth]{Graphics/HVAC_Diagram_Fans.pdf} % first figure itself
		\caption{Illustration of two stage refrigeration circuit}
		\label{fig:HVAC_Diagram}
	\end{minipage}\hfill
	\begin{minipage}{0.4\textwidth}
		\centering
		\includegraphics[width=1.05\textwidth]{Graphics/Flash_Tank_P-h_Diagram} % second figure itself
		\caption{Illustration of p-h diagram of two stage refrigeration circuit}
		\label{fig:p-h_diagram}
	\end{minipage}
\end{figure}

In \cref{fig:p-h_diagram} an illustrative p-h diagram of the refrigeration cicuit can be seen with the numbers referring to \cref{fig:HVAC_Diagram}. Two loops can be observed in \cref{fig:HVAC_Diagram}. The first is the outer loop ($1 \rightarrow 2\rightarrow 3 \rightarrow 4 \rightarrow 5 \rightarrow 6 \rightarrow 7 \rightarrow 8 \rightarrow 1$). This is considered the primary loop. The second loop is the inner right loop ($3 \rightarrow 4 \rightarrow 5 \rightarrow 6 \rightarrow 9 \rightarrow 3$). \\
The refrigeration system essentially works as a single stage system in the primary loop, where the compression and expansion is distributed along two stages, and as mentioned, the secondary loop  increases the efficiency of cooling. A description of the steps in the refrigeration cycle follows:
\begin{itemize}
	\item From point 1 to 2, the low pressure, high enthalpy vapor refrigerant from the evaporator is compressed. The output vapor refrigerant is at medium pressure and medium enthalpy. 
	\item From point 2 and 9 to 3, the compressor stage 1 output refrigerant is mixed with the flash gas from the flash tank. 
	\item From point 3 to 4, the medium pressure and enthalpy vapor refrigerant mix is compressed further, outputting high pressure and enthalpy vapor refrigerant.
	\item From point 4 to 5, the high pressure and enthalpy vapor refrigerant is condensed through the condenser, which decreases the enthalpy. This is where the heat from the load is transferred to the ambient air (hot body).
	\item From point 5 to 6, the pressure is decreased of the high pressure, low enthalpy liquid refrigerant. This results in a medium pressure, low enthalpy refrigerant where some of the refrigerant has changed phase to gas.
	\item From point 6 to 7 and 9 the medium pressure, low enthalpy liquid vapor mix will split into two mass flows. One is vapor, which goes to point 9, and the other which is liquid, goes to point 7. As the mix is split into two, so is the enthalpy. In particular liquid refrigerant going to 7 has lower enthalpy than the mix. This enables higher cooling capacity and increases the efficiency.
	\item From point 7 to 8, the liquid refrigerant is passed through an isenthalpic expansion which decreases the pressure and temperature.
	\item From point 8 to 1, the low pressure, low enthalpy liquid refrigerant extracts heat from the load by evaporating the refrigerant and thus encreases the enthalpy of the refrigerant.
\end{itemize}
Now that the principles of the refrigeration circuit is described, the load of system can be introduced. \\

\subsubsection{Refrigeration load}
The load for the BITZER refrigeration system is an insulated trailer containing cargo. The air inside the trailer is also part of the load, as is trailer box itself. The air inside the trailer is cooled via the evaporator of the refrigeration system and the air mass flow rate is driven by the evaporator fan. The cooled air extracts heat from the cargo and box, which is then transferred to the ambient air via the condenser. The heat flows and airflows from and to the trailer are illustrated in \cref{fig:trailer_airflow}.\\
The refrigeration system is defined and the control specific inputs and outputs can now be investigated.\\

\subsubsection{Control inputs}
Of the components seen in \cref{fig:HVAC_Diagram}, the two fans, the two valves and the compressor can be controlled. This yields a system that has 5 control inputs which manipulates the system behavior. The control inputs are:
\begin{itemize}
	\item The compressor speed $ \omega $ affects the flow of refrigerant in the system.
	\item The condenser fan speed $ U_{fan1} $ affects the heat transfer coefficient from the condenser to the ambient air.
	\item The evaporator fan speed  $ U_{fan2} $ affects the heat transfer coefficient from the circulated air to the evaporator. $ U_{fan2} $ additionally affects the circulating air flow, and adds heat to the air from its own power consumption.
	\item The condenser throttling valve opening degree $ \Theta_1 $ affects the equivalent flow resistance of the valve.
	\item The expansion valve opening degree $ \Theta_2 $ affects the equivalent flow resistance of the valve.
\end{itemize}

\subsubsection{Exogenous inputs}
In addition to the described control inputs, the system also has an exogenous input. 
Exogenous inputs are the inputs of a system that can't be controlled. They are commonly referred to as disturbances. The refrigeration system has one such disturbance, namely the ambient air temperature. The ambient temperature affects the heat transfer from ambient air to the air inside the reefer trailer and it also affects the condenser heat transfer.

\subsubsection{Controlled outputs}
There are several variables such as pressure and temperature throughout the refrigeration cycle, that are measured and hence are outputs of the system. The controlled outputs however, are those of interest from a control point of view. These are the outputs that the control strategy seeks to keep at a set point. In this project it will be the trailer box air temperature and the evaporator superheat. The superheat is the temperature difference between the refrigerant leaving the evaporator and the dew point temperature of the refrigerant. The output that is base for control of the superheat is such the temperature of refrigerant leaving the evaporator.
Controlling the air temperature is nessesary to ensure proper storage conditions for the cargo. Regulating the superheat is a more complex objective as it imposes two conflicting objectives. Having a large amount of superheat is wasteful and decreases the overall energy efficiency of the trailer. There should however be some amount of superheat, to ensure no liquid refrigerant enters the compressor, since that can be destructive to mechanical parts of the compressor.

\subsection{High-Fidelity model}
When working with any large scale system, iterative testing on the physical system is impractical. Slow dynamics and inconvienient test procedures make the proces too tedious for the user, and there is also a risk of equipment damage. Therefore, it is common practice to develop a High-Fidelity model of the system. Testing on a Hi-Fi model can be performed faster, efficiently and with no equipment risk. \\

BITZER has created such a Hi-Fi model for the cooling trailer in MATLAB/Simulink. The model has been shown to be highly accurate and can be trusted to mimic the real system behavior in most use-cases. The Hi-Fi model simulates the refrigeration system in \cref{fig:HVAC_Diagram}. The controlled inputs are regulated by a decoupled PID network. The cargo hold air temperature is regulated to -5$^{\circ}$C.\\

The control strategy developed in this project, will first and foremost be tested on the Hi-Fi model. Furthermore coefficients from the Hi-Fi model is extracted to enable modeling of the system components.


\subsection{Problem structure}
The overall goal of this project is to design and test a MIMO controller for refrigeration system. The controller objective is to keep the temperature inside the trailer constant despite exogenous inputs (disturbances), specifically the ambient temperature. 
Furthermore the controller should do so with the sub-goal of minimizing the energy consumption.

The controller will require access to relevant system information. In the context of state space control, these are the values of the system states. The states will be discussed in details later in the report. The states are used for feedback when choosing the proper control signals. Since the states of complex systems are usually not directly measurable, a simpler model will be developed for state observation. The model is then to be used for controller design. \\

The control model will be developed by means of first principle modeling. Doing so yields a non-linear model, due to the non-linear physical properties of thermodynamic systems. To apply linear control strategies, the model must be linearised. Optimal control strategies will be implemented on the linear model, as the control objectives can be well formulated as an optimization problem. The optimization problem being to keep the controlled outputs fixed, while minimizing the energy consumption of the system.\\

The next step in this project will be the modeling of the individual components of the refrigeration system.




		
