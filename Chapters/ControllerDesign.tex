This chapter contains the controller design for the refrigeration system model developed in \cref{sec:mod}. A simple pole-placement method controller will be made as an initial step to verify that the model is sufficient and to get a feel for system before moving on to a more complex controller. An MPC controller is chosen due to its ability to handle constraints which is essential for handling the actuator limitations in a reefer system. Both the pole-placement and MPC controller will be benchmarked against the coupled PID controllers currently implemented at BITZER.


\subsection{State space controller - Pole placement}
The poles of a controllable state space system is the eigenvalues of the system matrix $(A)$. When the states of a state space system is fed back through a feedback gain $(K)$ the system matrix of the closed-loop system becomes:

\begin{equation} \label{eq:A_cl}
	A_{cl} = A-BK
\end{equation}

If \cref{fig:state_space_fb} is analyzed by looking at the how $\dot{x}$ is calculated from $x$ this becomes obvious. A controllable system has the appealing property of  pole placement. This means that in theory the poles of the system can be moved anywhere to achieve some specific behavior. Such a change could be to make the system faster by placing them further to the left in the complex plane.

For a \underline{SISO} system the poles can be placed by hand from the following algorithm:
\begin{enumerate}
	\item defining desired pole placements for all system poles and writing out the characteristic polynomial for these wanted system poles e.g.
	$(s-p_1)(s-p_2) \cdots (s-p_n) = s^n + a_{cl_1} \cdot s^{n-1} + a_{cl_2} \cdot s^{n-2} \cdots a_{cl_n}$
	\item calculating the eigenvalues of \cref{eq:A_cl} which yields the characteristic polynomial of the feedback system e.g.
	$det(A_{sys}) = s^n + (a_{sys_1}-k_1) \cdot s^{n-1} + (a_{sys_2}-k_2) \cdot s^{n-2} \cdots (a_{sys_n}-k_n)$
	\item setting the coefficients of the characteristic polynomial of the native system equal to the coefficients of the desired pole placement and solving for the entries in K e.g.
	$ k_1 = a_{sys_1}-a_{cl_1}, k_2 = a_{sys_2}-a_{cl_2} \cdots k_n = a_{sys_n}-a_{cl_n} $
	 This yields the $ K = [k_1, k_2 \cdots k_n]^T $ which places the poles at the desired location.
	\item x is fed back through K to the input yielding \cref{eq:A_cl}.
\end{enumerate}

In Matlab the above algorithm can be performed by the simple function \textit{Place(A, B, poles)} where "A" is the system matrix, "B" the input matrix and "poles" an array of the desired poles. The \textit{place()} works for both SISO and MIMO systems.







\subsection{LMI / MPC controller}
