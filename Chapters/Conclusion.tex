%\textbf{THE GOOD}

%\begin{itemize}
%	\item \sout{Stable linear model was successfully made from separate component models and used to control Hi-Fi model close to operating point.}
%	\item \sout{Unobservable states succesfully removed yielding fully controllable and observable model.}
%\end{itemize}

%\noindent \textbf{THE BAD}

%\begin{itemize}
%	\item \sout{Mpjj Mcon are faulty states.}
%	\item \sout{Initial linerisation yielded a zero pole - not a good linearisation point.}
%	\item Condition number of controllability matrix might hint at poor controllability
%	\item \sout{Removal of Mpjj and Mcon yields linearised model with only three usable inputs}.
%	\item \sout{Kalman decomposition yields linearised model with only two usable inputs.}
%	\item Poor disturbance rejection due to observer structure in Hi-Fi simulation test.
%	\item Goal not reached
%\end{itemize}

%\noindent \textbf{THE GOOD \#2}

%\begin{itemize}
%	\item Kalman decomposition reduced model successfully estimates both full linear model states and Hi-Fi model states Tair and Tv.
%\end{itemize}

%\textbf{Conclusion text:} \\
The underlying physics of a trailer refrigeration system are complex, as several components of the system require extensive modeling to accurately represent reality. In the context of control the modeling task becomes to figure out what variables, states and equations are \textit{sufficient} for controller development and state observation. In this report a model based largely on algebraic equations was developed, only defining states to capture dynamics where large thermal capacitance resided. A stable, linear model was found at an equilibrium point and two unobservable states were subsequently removed with the Kalman decomposition. This left a fully controllable and observable model which was successfully used to control the supplied reefer trailer Hi-Fi simulation model, near the equilibrium point.\\

"\textit{All models are wrong, but some are useful}" is a famous aphorism by statistician George Box. It is useful to keep in mind when developing models, that they do not need to describe the real world perfectly, as long they are useful for their intended purpose. The control model developed in this project is not without its faults. A simulation of the non-linear model showed inaccuracies with regards to the behavior of the refrigerant masses inside the condenser ($M_{con}$) and the pipe joining junction ($M_{pjj}$). Closer analysis revealed insufficient connection between the component models. Variables such as pressure, temperature and enthalpy, which are outputs of some components, are not always passed correctly to adjacent components, that need them as inputs. Rather, they were fed predefined steady state values. In the best of cases, this just added to the model inaccuracies, making the model less accurate away from the linearisation point. It can however, result in issues as seen with the states $M_{con}$ and $M_{pjj}$, where the mass flow in and out of a control volume does not converge. This cause a continuous accumulation of mass, which is clearly not physically possible. \\

Although the chosen operating point yielded a stable linear model, one eigenvalue had a real part very close to zero. Per the Hartman-Grobman Theorem this implies that there is some of the dynamical behavior of the non-linear model, which is not being included in the linear model. Further, the the removal of the two faulty states resulted in a linear model only containing three inputs which mapped to the remaining states. The Kalman decomposition further removed an input, leaving only two usable inputs, namely the expansion valve opening degree $\Theta_1$ and the evaporator fan speed $U_{fan2}$. The optimal Infinite Horizon Linear Quadratic Regulator was implemented due to the intuitive way costs can be assigned to states and inputs. Subsequently a Luenberger observer gain was calculated based on the poles of the closed loop system, to ensure that it was faster than the controlled system. Tests on the Kalman decomposed reduced linear model proved that it was adequate as an observer by successfully estimating the linear model's observable states.\\

Despite the inadequacies of the resulting linearised model the controller and observer were still able to stabilize the Hi-Fi reefer trailer simulation system around the operating point. Furthermore it proved to be an accurate observer, yielding a very small error on the states $T_v$ and $T_{air}$. This result in and of itself is promising for future work. The derivation of a low order model, which is able to describe much of the important dynamics of the real system, is rather useful in a control problem. Previous work has derived control models with upwards of 80 states, which can pose computational difficulties when developing MIMO controllers.\\

%Furthermore, the control strategy implemented in this project relies only on measurements of two variables, namely the cargo hold air temperature and the temperature of the refrigerant in the compressor suction line.  Further development of this control strategy might allow for a cooling system with fewer sensors than is currently being used. This would reduce the building cost of the cooling system, which is always desirable for a business. \\

%Furthermore, the reduced amount of sensors increases the reliability of the system, as there are fewer components that can fail. The reduced amount of sensors required might even allow for installation of duplicate sensors, which would further increase reliability. This would also add increased accuracy on the measurement, as noise is inherently filtered by using multiple measurements.\\

The results concluded that the LQR controller is stable, even under the presence of a sine- or step disturbance in the ambient temperature. The control strategy, however, showed poor disturbance rejection, which is a common issue with standard LQR controllers. The system was further shown to be robustly stable with respect to a reasonable deviation in cargo heat transfer coefficient. The tests concluded that the energy consumption was approximately the same as the currently used PID structure, and such the developed controller did not improve efficiency of the refrigeration system.\\


















