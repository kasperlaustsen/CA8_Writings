A general state space system is typically defined on the form in \cref{eq:state_space}. It is observed that a change in states over time is defined as a linear combination of vectors and matrices. Having a model on such a state space form is a prerequisite for state-space controller design.

\begin{equation} \label{eq:state_space}
	\begin{split}
		\dot{\textbf{x}} & = \textbf{A}\textbf{x} + \textbf{B}\textbf{u} + \textbf{B}_d\textbf{d} \\
		\textbf{y} 		& = \textbf{C}\textbf{x}
	\end{split}
\end{equation}

where


\begin{center}
	\begin{tabular}{l p{8cm} l}
		$\textbf{x}$       & State vector                    &  \\
		$\dot{\textbf{x}}$ & Time derivative of state vector &  \\
		$\textbf{y}$       & Output vector                   &  \\
		$\textbf{d}$       & Disturbance vector              &  \\
		$\textbf{A}$       & System matrix                   &  \\
		$\textbf{B}$       & Controllable input matrix       &  \\
		$\textbf{B}_d$     & Disturbance input matrix        &  \\
		$\textbf{C}$       & Output matrix                   &
	\end{tabular}
\end{center}

A prerequisite for making a controller is that the system model used for making the controller is linear. This chapter examines the linearisation of the non-linear model found in \cref{sec:mod}. Linearisation is achieved through a first order Taylor expansion.

Consider some system $\dot{\textbf{x}} = \textbf{f}(\textbf{x},\textbf{u},\textbf{d})$ with \textbf{x} a vector of system states, \textbf{u} a vector of controlled inputs, and \textbf{d} a vector of disturbances. Such a system can be approximated with a first order taylor expansion at an operating point ($\textbf{x}_o, \textbf{u}_o, \textbf{d}_o$) as such:




\begin{equation} \label{eq:taylor}
	\dot{\textbf{x}}   \approx   \textbf{f}(\textbf{x}_o, \textbf{u}_o, \textbf{d}_o)   +   
	\left. \dfrac{\partial \textbf{f}}{\partial \textbf{x}} \right |_{\textbf{x}_o, \textbf{u}_o, \textbf{d}_o} \cdot (\textbf{x}-\textbf{x}_0) + 
	\left. \dfrac{\partial \textbf{f}}{\partial \textbf{u}} \right |_{\textbf{x}_o, \textbf{u}_o, \textbf{d}_o} \cdot (\textbf{u}-\textbf{u}_0) + 
	\left. \dfrac{\partial \textbf{f}}{\partial \textbf{d}} \right |_{\textbf{x}_o, \textbf{u}_o, \textbf{d}_o} \cdot (\textbf{d}-\textbf{d}_0)
\end{equation}

In \cref{eq:taylor} $f(\textbf{x}_o, \textbf{u}_o, \textbf{d}_o) = 0$ because the linearisation is done at an equilibrium point. The partial derivatives can be organised in Jacobian matrices on the form seen in \cref{eq:jacobian_states} where e is the number of state equations in the non linear system, and n is the number of variables to be differentiated with respect to, (e.g. number of states, iputs and disturbances). 

\begin{equation} \label{eq:jacobian_states}
	\dfrac{\partial \textbf{f}}{\partial \textbf{x}} =
		\begin{bmatrix}
			\dfrac{\partial f_1}{\partial x_1} & \cdots & \dfrac{\partial f_1}{\partial x_n} & \\
			\vdots & \ddots & \vdots & \\
			\dfrac{\partial f_e}{\partial x_1} & \cdots & \dfrac{\partial f_e}{\partial x_n} &
		\end{bmatrix}, \
	\dfrac{\partial \textbf{f}}{\partial \textbf{u}} = 	
		\begin{bmatrix}
			\dfrac{\partial f_1}{\partial u_1} & \cdots & \dfrac{\partial f_1}{\partial u_n} & \\
			\vdots & \ddots & \vdots & \\
			\dfrac{\partial f_e}{\partial u_1} & \cdots & \dfrac{\partial f_e}{\partial u_n} &
		\end{bmatrix}, \ 
	\dfrac{\partial \textbf{f}}{\partial \textbf{d}} = 	
		\begin{bmatrix}
			\dfrac{\partial f_1}{\partial d_1} & \cdots & \dfrac{\partial f_1}{\partial d_n} & \\
			\vdots & \ddots & \vdots & \\
			\dfrac{\partial f_e}{\partial d_1} & \cdots & \dfrac{\partial f_e}{\partial d_n} &
		\end{bmatrix}
\end{equation}


%\begin{equation} \label{eq:jacobian_inputs}
%	\dfrac{\partial \textbf{f}}{\partial \textbf{u}} = 	\begin{bmatrix}
%	\dfrac{\partial f_1}{\partial u_1} & \cdots & \dfrac{\partial f_1}{\partial u_n} & \\
%	\vdots & \ddots & \vdots & \\
%	\dfrac{\partial f_e}{\partial u_1} & \cdots & \dfrac{\partial f_e}{\partial u_n} &
%	\end{bmatrix}, \ 
%\end{equation}
%
%\begin{equation} \label{eq:jacobian_disturbances}
%		\dfrac{\partial \textbf{f}}{\partial \textbf{d}} = 	\begin{bmatrix}
%		\dfrac{\partial f_1}{\partial d_1} & \cdots & \dfrac{\partial f_1}{\partial d_n} & \\
%		\vdots & \ddots & \vdots & \\
%		\dfrac{\partial f_e}{\partial d_1} & \cdots & \dfrac{\partial f_e}{\partial d_n} &
%		\end{bmatrix}
%\end{equation}

When the Jacobian matrices are evaluated at the operating point, they in fact become the matrices $ \textbf{A} $, $ \textbf{B} $ and $ \textbf{B}_d  $ of the linear system

\begin{equation}
	\left. \dfrac{\partial \textbf{f}}{\partial \textbf{x}} \right |_{\textbf{x}_o, \textbf{u}_o, \textbf{d}_o} = \textbf{A}, \
	\left. \dfrac{\partial \textbf{f}}{\partial \textbf{u}} \right |_{\textbf{x}_o, \textbf{u}_o, \textbf{d}_o} = \textbf{B}, \
	\left. \dfrac{\partial \textbf{f}}{\partial \textbf{d}} \right |_{\textbf{x}_o, \textbf{u}_o, \textbf{d}_o} = \textbf{B}_d
\end{equation}

The only difference from a standard state space system such en

\textbf{Simplification}
To model the behaviour of the simulation model, we do some simplifications with respect to the real system. 
\begin{enumerate}
	\item valve is modeled as linear type
	\item coefficient XX is modelled as..
	\item
	\item
	\item
	\item
	
\end{enumerate}