A general state space system is typically defined on the form in \cref{eq:state_space}. It is observed that a change in states over time is defined as a linear combination of vectors and matrices. Having a model on such a state space form is a prerequisite for state-space controller design.

\begin{equation} \label{eq:state_space}
	\begin{split}
		\dot{x} & = Ax + Bu + B_dd \\
		y 		& = Cx
	\end{split}
\end{equation}

where


\begin{center}
	\begin{tabular}{l p{8cm} l}
		$x$       & State vector                    &  \\
		$\dot{x}$ & Time derivative of state vector &  \\
		$y$       & Output vector                   &  \\
		$d$       & Disturbance vector              &  \\
		$A$       & System matrix                   &  \\
		$B$       & Controllable input matrix       &  \\
		$B_d$     & Disturbance input matrix        &  \\
		$C$       & Output matrix                   &
	\end{tabular}
\end{center}

A prerequisite for making a controller is that the system model used for making the controller is linear. This chapter examines the linearisation of the non-linear model found in \cref{sec:mod}. Linearisation is achieved through a first order Taylor expansion.

Consider some system $\dot{\textbf{x}} = f(\textbf{x},\textbf{u},\textbf{d})$ with \textbf{x} a vector of system states, \textbf{u} a vector of controlled inputs, and \textbf{d} a vector of disturbances. Such a system can be approximated with a first order taylor expansion at an operating point ($\textbf{x}_o, \textbf{u}_o, \textbf{d}_o$) as such:




\begin{equation} \label{eq:taylor}
	\dot{\textbf{x}}   \approx   f(\textbf{x}_o, \textbf{u}_o, \textbf{d}_o)   +   \left. \dfrac{\partial f}{\partial \textbf{x}} \right |_{\textbf{x}_o, \textbf{u}_o, \textbf{d}_o} + \left. \dfrac{\partial f}{\partial \textbf{u}} \right |_{\textbf{x}_o, \textbf{u}_o, \textbf{d}_o} + \left. \dfrac{\partial f}{\partial \textbf{d}} \right |_{\textbf{x}_o, \textbf{u}_o, \textbf{d}_o}
\end{equation}

In \cref{eq:taylor} $f(\textbf{x}_o, \textbf{u}_o, \textbf{d}_o) = 0$ because the linearisation is done at an equilibrium point. The partial derivatives become Jacobian matrices on the form seen in \cref{eq:jacobian_general} where n is the number of states and e the number of equations.

\begin{equation} \label{eq:jacobian_general}
	\nabla f = 	\begin{bmatrix}
					\dfrac{\partial f_1}{\partial x_1} & \cdots & \dfrac{\partial f_1}{\partial x_n} & \\
					\vdots & \ddots & \vdots & \\
					\dfrac{\partial f_e}{\partial x_1} & \cdots & \dfrac{\partial f_e}{\partial x_n} &
				\end{bmatrix}
\end{equation}



\begin{equation} \label{eq:}
	1=1
\end{equation}

\begin{equation} \label{eq:}
	1=1
\end{equation}


\textbf{Simplification}
To model the behaviour of the simulation model, we do some simplifications with respect to the real system. 
\begin{enumerate}
	\item valve is modeled as linear type
	\item coefficient XX is modelled as..
	\item
	\item
	\item
	\item
	
\end{enumerate}